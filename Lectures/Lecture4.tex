\PassOptionsToPackage{unicode=true}{hyperref} % options for packages loaded elsewhere
\PassOptionsToPackage{hyphens}{url}
\PassOptionsToPackage{dvipsnames,svgnames*,x11names*}{xcolor}
%
\documentclass[ignorenonframetext,]{beamer}
\usepackage{pgfpages}
\setbeamertemplate{caption}[numbered]
\setbeamertemplate{caption label separator}{: }
\setbeamercolor{caption name}{fg=normal text.fg}
\beamertemplatenavigationsymbolsempty
% Prevent slide breaks in the middle of a paragraph:
\widowpenalties 1 10000
\raggedbottom
\setbeamertemplate{part page}{
\centering
\begin{beamercolorbox}[sep=16pt,center]{part title}
  \usebeamerfont{part title}\insertpart\par
\end{beamercolorbox}
}
\setbeamertemplate{section page}{
\centering
\begin{beamercolorbox}[sep=12pt,center]{part title}
  \usebeamerfont{section title}\insertsection\par
\end{beamercolorbox}
}
\setbeamertemplate{subsection page}{
\centering
\begin{beamercolorbox}[sep=8pt,center]{part title}
  \usebeamerfont{subsection title}\insertsubsection\par
\end{beamercolorbox}
}
\AtBeginPart{
  \frame{\partpage}
}
\AtBeginSection{
  \ifbibliography
  \else
    \frame{\sectionpage}
  \fi
}
\AtBeginSubsection{
  \frame{\subsectionpage}
}
\usepackage{lmodern}
\usepackage{amssymb,amsmath}
\usepackage{ifxetex,ifluatex}
\usepackage{fixltx2e} % provides \textsubscript
\ifnum 0\ifxetex 1\fi\ifluatex 1\fi=0 % if pdftex
  \usepackage[T1]{fontenc}
  \usepackage[utf8]{inputenc}
  \usepackage{textcomp} % provides euro and other symbols
\else % if luatex or xelatex
  \usepackage{unicode-math}
  \defaultfontfeatures{Ligatures=TeX,Scale=MatchLowercase}
\fi
% use upquote if available, for straight quotes in verbatim environments
\IfFileExists{upquote.sty}{\usepackage{upquote}}{}
% use microtype if available
\IfFileExists{microtype.sty}{%
\usepackage[]{microtype}
\UseMicrotypeSet[protrusion]{basicmath} % disable protrusion for tt fonts
}{}
\IfFileExists{parskip.sty}{%
\usepackage{parskip}
}{% else
\setlength{\parindent}{0pt}
\setlength{\parskip}{6pt plus 2pt minus 1pt}
}
\usepackage{xcolor}
\usepackage{hyperref}
\hypersetup{
            pdftitle={Lecture 4 - MI with derived variables, survival outcomes, dependent data and survey data},
            pdfauthor={Jonathan Bartlett (thestatsgeek.com)},
            colorlinks=true,
            linkcolor=blue,
            filecolor=Maroon,
            citecolor=Blue,
            urlcolor=blue,
            breaklinks=true}
\urlstyle{same}  % don't use monospace font for urls
\newif\ifbibliography
\usepackage{longtable,booktabs}
\usepackage{caption}
% These lines are needed to make table captions work with longtable:
\makeatletter
\def\fnum@table{\tablename~\thetable}
\makeatother
\setlength{\emergencystretch}{3em}  % prevent overfull lines
\providecommand{\tightlist}{%
  \setlength{\itemsep}{0pt}\setlength{\parskip}{0pt}}
\setcounter{secnumdepth}{0}

% set default figure placement to htbp
\makeatletter
\def\fps@figure{htbp}
\makeatother


\title{Lecture 4 - MI with derived variables, survival outcomes, dependent data
and survey data}
\providecommand{\subtitle}[1]{}
\subtitle{Multiple imputation for missing data}
\author{\href{https://thestatsgeek.com}{Jonathan Bartlett (thestatsgeek.com)}}
\date{Oslo, November 2019}

\begin{document}
\frame{\titlepage}

\begin{frame}
\tableofcontents[hideallsubsections]
\end{frame}
\hypertarget{derived-variables}{%
\section{Derived variables}\label{derived-variables}}

\begin{frame}[fragile]{Derived variables}
\protect\hypertarget{derived-variables-1}{}

\begin{itemize}
\tightlist
\item
  If our substantive model of interest includes derived variables, like
  non-linear effects and/or interactions, our imputation model should
  respect these.
\item
  In the next practical, we will look at a substantive model for
  \texttt{sbp} in NHANES which includes the interaction between
  \texttt{waist\_circum} and \texttt{ALQ150}.
\item
  The imputation model should be `compatible' or congenial with the
  substantive model (Meng \protect\hyperlink{ref-Meng:1994}{1994}).
\item
  e.g.~suppose our model of interest is a linear regression of \(Y\) on
  \(X\) and \(X^2\), if we impute missing values of \(X\) using a linear
  regression of \(X\) on \(Y\), the imputed data will not have the
  correct quadratic relationship between \(Y\) and \(X\).
\end{itemize}

\end{frame}

\begin{frame}[fragile]{Interactions}
\protect\hypertarget{interactions}{}

\begin{itemize}
\tightlist
\item
  Suppose the outcome/substantive model contains an interaction between
  two predictors, \(X_{1}\) and \(X_{2}\), one of which (\(X_{1}\)) is
  categorical (e.g. \texttt{ALQ150}).
\item
  If \(X_{1}\) is fully observed, a convenient approach to allow for the
  interaction is to impute separately in different levels of \(X_{1}\).
\item
  Stata's \texttt{mi} commands make this very easy: simply add
  \texttt{by(x1)} at the end of the command.
\item
  In R, we could split the data into multiple data frames, run
  \texttt{mice} on each, and then recombining the imputed datasets.
\item
  If \(X_{1}\) itself has missing values (as in \texttt{ALQ150}
  variable), we cannot use this approach.
\item
  It also does not work if both \(X_{1}\) and \(X_{2}\) are continuous,
  or we want to allow for multiple interactions.
\end{itemize}

\end{frame}

\begin{frame}{Impute then transform}
\protect\hypertarget{impute-then-transform}{}

\begin{itemize}
\tightlist
\item
  The simplest approach to handling derived variables is to perform
  imputation as normal, then create the derived variables
  (e.g.~interactions) in the imputed datasets.
\item
  This is not a good idea.
\item
  The imputation models will not be compatible with what the substantive
  model.
\item
  Biased estimates will be obtained.
\end{itemize}

\end{frame}

\begin{frame}[fragile]{Passive imputation}
\protect\hypertarget{passive-imputation}{}

\begin{itemize}
\tightlist
\item
  Passive imputation involves adding the derived variable(s) to the data
  frame and updating its value during the imputation process.
\item
  e.g.~we can add a variable \texttt{waist\_circum*AQL150} and tell
  \texttt{mice} how to update its value.
\item
  This interaction term can be used as a covariate in the imputation
  models.
\end{itemize}

\end{frame}

\begin{frame}[fragile]{Limitations of passive imputation}
\protect\hypertarget{limitations-of-passive-imputation}{}

\begin{itemize}
\tightlist
\item
  Passive imputation has limitations - it is not always obvious how to
  specify imputation models which are compatible with the substantive
  model.
\item
  e.g.~when imputing \texttt{ALQ150}, we need to ensure its imputation
  model is compatible with the presence of an interaction between it and
  \texttt{waist\_circum} in the model for \texttt{sbp}.
\item
  Used naively it will usually lead to biased estimates.
\end{itemize}

\end{frame}

\begin{frame}[fragile]{`Just another variable' approach}
\protect\hypertarget{just-another-variable-approach}{}

\begin{itemize}
\tightlist
\item
  The `transform then impute' or `just another variable' (JAV) approach
  recently proposed by von Hippel (Hippel
  \protect\hyperlink{ref-Hippel2009}{2009}) involves treating derived
  variables as if they were just any other variables and includes them
  in the imputation process.
\item
  e.g.~we include \texttt{waist\_circum*AQL150} in the imputation
  process and impute it as if it were a regular continuous variable, and
  ignore the deterministic relationship between it and
  \texttt{waist\_circum} and \texttt{ALQ150}.
\item
  An unappealing feature of this is that we have imputed values of
  \texttt{waist\_circum*AQL150} which are not equal to the product of
  the values of \texttt{waist\_circum} and \texttt{ALQ150}.
\end{itemize}

\end{frame}

\begin{frame}{Statistical properties of the `just another variable'
approach}
\protect\hypertarget{statistical-properties-of-the-just-another-variable-approach}{}

\begin{itemize}
\tightlist
\item
  For linear models where data are MCAR, the JAV approach gives
  consistent point estimates, but Rubin's rules may not be valid.
\item
  With data MAR, JAV gives biased estimates, since it consists of
  fitting a mis-specified parametric model by maximum likelihood.
\item
  For logistic regression models JAV can be badly biased.
\item
  For more on this, see (S. R. Seaman, Bartlett, and White
  \protect\hyperlink{ref-Seaman2012}{2012}).
\end{itemize}

\end{frame}

\begin{frame}{Substantive model compatible FCS (SMC-FCS)}
\protect\hypertarget{substantive-model-compatible-fcs-smc-fcs}{}

\begin{itemize}
\tightlist
\item
  We developed a modified version of MICE/FCS, which imputes each
  covariate compatibly with a user-specified substantive model (SM)
  (Bartlett et al. \protect\hyperlink{ref-Bartlett2014}{2015}).
\item
  Suppose we have an outcome of interest \(Y\), partially observed
  covariates \(X_{1},X_{2},..,X_{p}\), and fully observed covariates
  \(\mathbf Z\).
\item
  We specify a substantive model (SM) for
  \(f(Y|X_{1},..,X_{p},\mathbf Z,\psi)\), with parameters \(\psi\).
\item
  e.g.~linear regression of \(Y\), with covariate vector some function
  of \(X_{1},..,X_{p}\) and \(\mathbf Z\).
\item
  e.g.~covariates include \(X_{1} \times X_{2}\), or \(X_{1}^2\), or
  \(X_{1}/X_{2}^2\)\ldots{}
\item
  The covariates \(X_{1},..,X_{p}\) have missing values.
\end{itemize}

\end{frame}

\begin{frame}{Substantive model compatible FCS}
\protect\hypertarget{substantive-model-compatible-fcs}{}

\begin{itemize}
\tightlist
\item
  We must impute from a model for \(f(X_{j}|X_{-j},\mathbf Z,Y)\).
\item
  This can be expressed as \begin{align*}
    \frac{f(Y|X_{j},X_{-j},\mathbf Z)f(X_{j}|X_{-j},\mathbf Z)}{\int f(Y|X^{*}_{j},X_{-j},\mathbf Z) f(X^{*}_{j}|X_{-j},\mathbf Z) dX^{*}_{j}}.
    \end{align*}
\item
  The SM is a model for \(f(Y|X_{j},X_{-j},\mathbf Z)\).
\item
  We can thus specify an IM for \(X_{j}\) which is compatible with the
  SM by additionally specifying a model for
  \(f(X_{j}|X_{-j},\mathbf Z)\).
\end{itemize}

\end{frame}

\begin{frame}{Drawing imputations}
\protect\hypertarget{drawing-imputations}{}

\begin{itemize}
\tightlist
\item
  Having specified a model for \(f(X_{j}|X_{-j},\mathbf Z)\), the
  implied imputation model \(f(X_{j}|X_{-j},\mathbf Z,Y)\) will in
  general not belong to a standard distributional family.
\item
  We appeal to the Monte-Carlo method of rejection sampling to generate
  draws.
\item
  Rejection sampling involves drawing from an easy-to-sample (candidate)
  distribution until a particular criterion/bound is satisfied.
\item
  Deriving this bound is relatively easy if we use our model for
  \(f(X_{j}|X_{-j},\mathbf Z)\) as the candidate distribution.
\end{itemize}

\end{frame}

\begin{frame}[fragile]{\texttt{smcfcs}}
\protect\hypertarget{smcfcs}{}

\begin{itemize}
\tightlist
\item
  \href{https://cran.r-project.org/package=smcfcs}{\texttt{smcfcs}}
  implements the SMC-FCS approach in R.
  \href{https://github.com/jwb133/Stata-smcfcs}{\texttt{smcfcs}} in
  Stata.
\item
  Linear, logistic and Cox proportional hazards outcome models are
  supported.
\item
  It also supports competing risks outcomes, and nested case-control and
  case-cohort studies.
\item
  Normal linear, logistic, Poisson, proportional odds and multinomial
  logistic imputation methods are provided.
\item
  The SM can contain essentially any function of the variables,
  e.g.~squares, cubes, interactions, logarithms of variables, etc etc.
\item
  The approach can also be used when imputing components of a ratio
  variable, e.g.~BMI.
\item
  In the practical we will see how \texttt{smcfcs} can be used to
  accommodate an interaction, and be used to impute missing covariates
  in a Cox model analysis.
\end{itemize}

\end{frame}

\hypertarget{survival-outcomes}{%
\section{Survival outcomes}\label{survival-outcomes}}

\begin{frame}{Incorporating the outcome in imputation}
\protect\hypertarget{incorporating-the-outcome-in-imputation}{}

\begin{itemize}
\tightlist
\item
  As we noted earlier, the outcome variable in the final model of
  interest \emph{must} be included in the imputation model.
\item
  If we do not, imputed values will not have the correct associations
  with the outcome.
\item
  How to incorporate the outcome in an imputation model depends on the
  type of variable being imputed and the type of outcome / outcome
  model.
\end{itemize}

\end{frame}

\begin{frame}{Survival outcomes}
\protect\hypertarget{survival-outcomes-1}{}

\begin{itemize}
\tightlist
\item
  A common outcome type is time to some event of interest (often called
  survival outcomes).
\item
  Sometimes we do not observe the event occurring for every subject in
  the available follow-up, leading to censoring.
\item
  The outcome then consists of a variable \(T\) representing time to the
  event of interest and an event indicator \(D\) (\(D=1\) if event
  occured, \(D=0\) otherwise).
\item
  If \(D=0\), \(T\) records the censoring time -- the last time at which
  a subject was seen, and had still not had the event.
\item
  If we have some missing values in the covariates \(X\) in our survival
  model, how should we impute them?
\end{itemize}

\end{frame}

\begin{frame}{Incorporating survival outcomes in imputation models}
\protect\hypertarget{incorporating-survival-outcomes-in-imputation-models}{}

\begin{itemize}
\tightlist
\item
  Early recommendations were to impute \(X\) by putting \(T\) (or
  \(\log(T)\) and \(D\) as covariates).
\item
  More recently, White and Royston
  (\protect\hyperlink{ref-White2009}{2009}) investigated theoretically
  how the imputation model for \(X\) should be specified when a Cox
  proportinal hazards model is used:\\
  \begin{align}
    h(t|\mathbf X) = h_{0}(t) \exp(\boldsymbol \beta^{T} \mathbf X)
  \end{align} where \(h_{0}(t)\) denotes an arbitrary baseline hazard
  function and \(\boldsymbol \beta\) a vector of (log) hazard ratios.
\end{itemize}

\end{frame}

\begin{frame}{Incorporating survival outcomes in imputation models}
\protect\hypertarget{incorporating-survival-outcomes-in-imputation-models-1}{}

\begin{itemize}
\tightlist
\item
  White and Royston showed that when imputing a normally distributed
  variable \(X\) one should use a linear regression imputation model,
  with \(D\) and \(H_{0}(T) = \int^{T}_{0} h_{0}(u) du\) (baseline
  cumulative hazard function) as covariates.
\item
  For binary \(X\), one should use a logistic regression imputation
  model, again with \(D\) and \(H_{0}(T)\) as covariates.
\item
  Their results are exact for binary \(X\), but are approximate for
  normal \(X\).
\item
  The approximation for normal \(X\) should work well provided the
  covariate \(X\) does not have a large effect on hazard or if the
  incidence of the event of interest is low.
\end{itemize}

\end{frame}

\begin{frame}{Incorporating survival outcomes in imputation models}
\protect\hypertarget{incorporating-survival-outcomes-in-imputation-models-2}{}

\begin{itemize}
\item
  \(H_{0}(t)=\int^{t}_{0} h_{0}(u) du\) is the baseline cumulative
  hazard function.
\item
  White and Royston suggest a number of approaches to estimating
  \(H_{0}(t)\):

  \begin{itemize}
  \tightlist
  \item
    Substantive knowledge - e.g.~it may be reasonable to assume constant
    baseline hazard so that \(H_{0}(t) \propto t\). In this case, we
    just include \(D\) and \(T\) as covariates in our imputation
    model(s).

    \begin{itemize}
    \tightlist
    \item
      When covariate effects are small, one could approximate
      \(H_{0}(t)\) by the Nelson-Aalen (marginal) cumulative hazard
      estimator \(H(t)\), which ignores covariates and thus can be
      estimated using all subjects.
    \item
      Estimating \(H_{0}(t)\) within the FCS algorithm by fitting the
      Cox proportional hazards model to the current imputed dataset.
    \end{itemize}
  \end{itemize}
\end{itemize}

\end{frame}

\begin{frame}{Substantive model compatible FCS}
\protect\hypertarget{substantive-model-compatible-fcs-1}{}

\begin{itemize}
\tightlist
\item
  Our SMC-FCS approach can also `solve' the problem.
\item
  Each partially observed covariate is imputed compatibly with the
  specified Cox model.
\item
  Our approach is particularly attractive if there are additionally
  interactions or non-linear covariate effects in the Cox model.
\item
  If censoring mechanism is related to partially observed covariate(s),
  then censoring should be treated as a competing risk at the imputation
  stage.
\end{itemize}

\end{frame}

\hypertarget{imputation-with-dependent-data}{%
\section{Imputation with dependent
data}\label{imputation-with-dependent-data}}

\begin{frame}{Example - longitudinal data}
\protect\hypertarget{example---longitudinal-data}{}

\begin{itemize}
\tightlist
\item
  Suppose some quantity \(y\) was intended to be measured repeatedly on
  subjects over time.
\item
  There are some missing values of \(y\).
\item
  How should we impute these missing values?
\end{itemize}

\end{frame}

\begin{frame}{Imputing in `long form'}
\protect\hypertarget{imputing-in-long-form}{}

\begin{longtable}[]{@{}llll@{}}
\toprule
id & gender & time & y\tabularnewline
\midrule
\endhead
1 & m & 0 & 4.5\tabularnewline
1 & m & 1 & 3.9\tabularnewline
1 & m & 2 & 4.1\tabularnewline
1 & m & 3 & .\tabularnewline
1 & m & 4 & 4.2\tabularnewline
\bottomrule
\end{longtable}

\begin{itemize}
\tightlist
\item
  If we impute the dataset is `long' form, we treat each observation as
  independent.
\item
  This is clearly inappropriate - observations from the same subject are
  usually correlated.
\item
  The observed values of \(y\) on a subject contain information about
  missing \(y\) at \(t=3\).
\item
  If we ignore the longitudinal structure, imputations will not only be
  inefficient, they will not have the correct correlation structure.
\end{itemize}

\end{frame}

\begin{frame}[fragile]{Imputing in `wide form'}
\protect\hypertarget{imputing-in-wide-form}{}

\begin{longtable}[]{@{}lllllll@{}}
\toprule
id & gender & y0 & y1 & y2 & y3 & y4\tabularnewline
\midrule
\endhead
1 & m & 4.5 & 3.9 & 4.1 & . & 4.2\tabularnewline
\bottomrule
\end{longtable}

\begin{itemize}
\tightlist
\item
  If measurement times are common to all subjects, we may be able to
  impute with the data in `wide' form.
\item
  e.g.~we could apply \texttt{mice} to gender, y0, y1, y2, y3, y4.
\item
  This uses available longitudinal information to impute missing value
  at \(t=3\) for \(\mbox{id}=1\).
\item
  Note that this strategy generally cannot be applied if observations
  take place at different times for different subjects.
\item
  You may run into co-linearity issues when \(y\) is highly correlated
  within subjects over time.
\end{itemize}

\end{frame}

\begin{frame}{Example - clustered data}
\protect\hypertarget{example---clustered-data}{}

\begin{itemize}
\tightlist
\item
  Another form of dependent data is clustered of multi-level data.
\item
  The clustering should be accounted for in the imputation process.
\end{itemize}

\end{frame}

\begin{frame}{Including fixed cluster effects}
\protect\hypertarget{including-fixed-cluster-effects}{}

\begin{itemize}
\tightlist
\item
  One approach is to include cluster id as a fixed effect covariate in
  imputation models.
\item
  Standard MI software can be used.
\item
  If each cluster has a large number of (observed) units, this could
  work well.
\item
  But if the substantive model is a random effects model, it has been
  shown to lead to invalid inferences (Andridge
  \protect\hyperlink{ref-Andridge2011}{2011}).
\end{itemize}

\end{frame}

\begin{frame}{Imputation by cluster}
\protect\hypertarget{imputation-by-cluster}{}

\begin{itemize}
\tightlist
\item
  Yet another approach is to impute separately in each cluster, thus
  allowing parameters to vary by cluster.
\item
  This should work well when there are a small number of large clusters.
\item
  An advantage of this approach is that it allows all the imputation
  model parameters to differ between clusters.
\item
  Conversely, a disadvantage is that information is not borrowed between
  clusters.
\item
  If there are many clusters, and/or clusters are small, imputation by
  cluster may perform poorly.
\end{itemize}

\end{frame}

\begin{frame}{Random-effects imputation}
\protect\hypertarget{random-effects-imputation}{}

\begin{itemize}
\tightlist
\item
  If your substantive analyses would treat the dependency in the data
  through random effects, you should probably impute mising data using
  random effects models.
\item
  The principles of MI remain the same - all that changes is that we
  have random effects in our imputation model(s).
\end{itemize}

\end{frame}

\begin{frame}[fragile]{Random-effects imputation software}
\protect\hypertarget{random-effects-imputation-software}{}

\begin{itemize}
\tightlist
\item
  \href{https://cran.r-project.org/package=mice}{\texttt{mice}} can
  impute variables using FCS with certain random effects models.
\item
  \href{https://cran.r-project.org/package=jomo}{\texttt{jomo}} can
  impute using joint random effects models based on latent multivariate
  normal structure.
\item
  \href{https://cran.r-project.org/package=JointAI}{\texttt{jointAI}}
  can impute using joint random effects models based on factorising the
  joint distribution as a product of conditionals.
\item
  For further details, see Buuren
  (\protect\hyperlink{ref-vanBuuren2011}{2011}) and Audigier et al.
  (\protect\hyperlink{ref-audigier2018multiple}{2018}).
\end{itemize}

\end{frame}

\begin{frame}{Likelihood based approaches}
\protect\hypertarget{likelihood-based-approaches}{}

\begin{itemize}
\tightlist
\item
  If missingness is confined to the outcome variable, likelihood based
  approaches are statistically efficient and valid under MAR.
\item
  In such cases, there is little point in trying to attempt imputation.
\item
  This is the case for both dependent data situations and simpler
  independent data situations.
\end{itemize}

\end{frame}

\hypertarget{survey-data}{%
\section{Survey data}\label{survey-data}}

\begin{frame}{MI with survey data}
\protect\hypertarget{mi-with-survey-data}{}

\begin{itemize}
\tightlist
\item
  Rubin's original aim was for MI to be used in the context of large
  survey datasets.
\item
  However, it is not clear how `proper' imputations can be generated
  when data were collected using a complex survey design.
\item
  S. R. Seaman et al. (\protect\hyperlink{ref-Seaman2012a}{2012})
  derived some important results concerning this.
\end{itemize}

\end{frame}

\begin{frame}{Recommendations for MI with survey data}
\protect\hypertarget{recommendations-for-mi-with-survey-data}{}

Seaman et al's results imply that when performing MI with weighted
survey data we should:

\begin{enumerate}
\tightlist
\item
  include the sample design weights as a covariate (not as a weight!) in
  the imputation model(s),
\item
  when analysing the imputed datasets, the completed data analyses
  should be weighted using the survey design weights.
\end{enumerate}

\end{frame}

\begin{frame}{Variance estimation}
\protect\hypertarget{variance-estimation}{}

\begin{itemize}
\tightlist
\item
  One issue is that Rubin's variance estimator can be biased upwards
  (conservative inference) if the imputer makes an assumption which the
  analyst doesn't.
\item
  In simple settings, Seaman et al showed that the upward bias in
  variance estimates could be avoided by including interactions between
  weights and fully observed variables.
\item
  Even when Rubin's variance estimator was biased (upwards), Seaman et
  al found that the bias was small.
\item
  In practice therefore, we might worry less about this issue.
\end{itemize}

\end{frame}

\hypertarget{summary}{%
\section{Summary}\label{summary}}

\begin{frame}{Summary}
\protect\hypertarget{summary-1}{}

\begin{itemize}
\tightlist
\item
  Care must be taken that variables are imputed compatibly/congenially
  with subsequent analyses (substantive models).
\item
  In particular, imputing derievd variables involved in interactions or
  non-linear effects requires care.
\item
  e.g.~with a Cox model, we must account for the outcome appropriately
  if imputing covariates.
\item
  e.g.~with dependent data, our imputation model should ideally account
  for the dependency.
\end{itemize}

\end{frame}

\begin{frame}[allowframebreaks]{References}
\protect\hypertarget{references}{}

\hypertarget{refs}{}
\leavevmode\hypertarget{ref-Andridge2011}{}%
Andridge, Rebecca R. 2011. ``Quantifying the Impact of Fixed Effects
Modeling of Clusters in Multiple Imputation for Cluster Randomized
Trials.'' \emph{Biometrical Journal} 53 (1). Wiley Online Library:
57--74.

\leavevmode\hypertarget{ref-audigier2018multiple}{}%
Audigier, Vincent, Ian R White, Shahab Jolani, Thomas PA Debray, Matteo
Quartagno, James Carpenter, Stef Van Buuren, Matthieu Resche-Rigon, and
others. 2018. ``Multiple Imputation for Multilevel Data with Continuous
and Binary Variables.'' \emph{Statistical Science} 33 (2). Institute of
Mathematical Statistics: 160--83.

\leavevmode\hypertarget{ref-Bartlett2014}{}%
Bartlett, J W, S R Seaman, I R White, and J R Carpenter. 2015.
``Multiple imputation of covariates by fully conditional specification:
Accommodating the substantive model.'' \emph{Statistical Methods in
Medical Research} 24 (4): 462--87.
\url{https://doi.org/10.1177/0962280214521348}.

\leavevmode\hypertarget{ref-vanBuuren2011}{}%
Buuren, S van. 2011. ``Multiple Imputation of Multilevel Data.'' In
\emph{The Handbook of Advanced Multilevel Analysis}, 173--96. Routledge.

\leavevmode\hypertarget{ref-Hippel2009}{}%
Hippel, P T von. 2009. ``How to Impute Interactions, Squares, and Other
Transformed Variables.'' \emph{Sociological Methodology} 39: 265--91.

\leavevmode\hypertarget{ref-Meng:1994}{}%
Meng, X L. 1994. ``Multiple-Imputation Inferences with Uncongenial
Sources of Input (with Discussion).'' \emph{Statistical Science} 10:
538--73.

\leavevmode\hypertarget{ref-Seaman2012}{}%
Seaman, S. R., J. W. Bartlett, and I. R. White. 2012. ``Multiple
imputation of missing covariates with non-linear effects and
interactions: an evaluation of statistical methods.'' \emph{BMC Medical
Research Methodology} 12: 46.

\leavevmode\hypertarget{ref-Seaman2012a}{}%
Seaman, S. R., I. R. White, A. J. Copas, and L. Li. 2012. ``Combining
Multiple Imputation and Inverse-Probability Weighting.''
\emph{Biometrics} 68: 129--37.

\leavevmode\hypertarget{ref-White2009}{}%
White, I. R., and P. Royston. 2009. ``Imputing missing covariate values
for the Cox model.'' \emph{Statistics in Medicine} 28: 1982--98.

\end{frame}

\end{document}
