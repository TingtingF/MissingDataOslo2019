\PassOptionsToPackage{unicode=true}{hyperref} % options for packages loaded elsewhere
\PassOptionsToPackage{hyphens}{url}
\PassOptionsToPackage{dvipsnames,svgnames*,x11names*}{xcolor}
%
\documentclass[ignorenonframetext,]{beamer}
\usepackage{pgfpages}
\setbeamertemplate{caption}[numbered]
\setbeamertemplate{caption label separator}{: }
\setbeamercolor{caption name}{fg=normal text.fg}
\beamertemplatenavigationsymbolsempty
% Prevent slide breaks in the middle of a paragraph:
\widowpenalties 1 10000
\raggedbottom
\setbeamertemplate{part page}{
\centering
\begin{beamercolorbox}[sep=16pt,center]{part title}
  \usebeamerfont{part title}\insertpart\par
\end{beamercolorbox}
}
\setbeamertemplate{section page}{
\centering
\begin{beamercolorbox}[sep=12pt,center]{part title}
  \usebeamerfont{section title}\insertsection\par
\end{beamercolorbox}
}
\setbeamertemplate{subsection page}{
\centering
\begin{beamercolorbox}[sep=8pt,center]{part title}
  \usebeamerfont{subsection title}\insertsubsection\par
\end{beamercolorbox}
}
\AtBeginPart{
  \frame{\partpage}
}
\AtBeginSection{
  \ifbibliography
  \else
    \frame{\sectionpage}
  \fi
}
\AtBeginSubsection{
  \frame{\subsectionpage}
}
\usepackage{lmodern}
\usepackage{amssymb,amsmath}
\usepackage{ifxetex,ifluatex}
\usepackage{fixltx2e} % provides \textsubscript
\ifnum 0\ifxetex 1\fi\ifluatex 1\fi=0 % if pdftex
  \usepackage[T1]{fontenc}
  \usepackage[utf8]{inputenc}
  \usepackage{textcomp} % provides euro and other symbols
\else % if luatex or xelatex
  \usepackage{unicode-math}
  \defaultfontfeatures{Ligatures=TeX,Scale=MatchLowercase}
\fi
% use upquote if available, for straight quotes in verbatim environments
\IfFileExists{upquote.sty}{\usepackage{upquote}}{}
% use microtype if available
\IfFileExists{microtype.sty}{%
\usepackage[]{microtype}
\UseMicrotypeSet[protrusion]{basicmath} % disable protrusion for tt fonts
}{}
\IfFileExists{parskip.sty}{%
\usepackage{parskip}
}{% else
\setlength{\parindent}{0pt}
\setlength{\parskip}{6pt plus 2pt minus 1pt}
}
\usepackage{xcolor}
\usepackage{hyperref}
\hypersetup{
            pdftitle={Lecture 3 - MI for multiple variables and MI practicalities},
            pdfauthor={Jonathan Bartlett (thestatsgeek.com)},
            colorlinks=true,
            linkcolor=blue,
            filecolor=Maroon,
            citecolor=Blue,
            urlcolor=blue,
            breaklinks=true}
\urlstyle{same}  % don't use monospace font for urls
\newif\ifbibliography
\usepackage{longtable,booktabs}
\usepackage{caption}
% These lines are needed to make table captions work with longtable:
\makeatletter
\def\fnum@table{\tablename~\thetable}
\makeatother
\setlength{\emergencystretch}{3em}  % prevent overfull lines
\providecommand{\tightlist}{%
  \setlength{\itemsep}{0pt}\setlength{\parskip}{0pt}}
\setcounter{secnumdepth}{0}

% set default figure placement to htbp
\makeatletter
\def\fps@figure{htbp}
\makeatother


\title{Lecture 3 - MI for multiple variables and MI practicalities}
\providecommand{\subtitle}[1]{}
\subtitle{Multiple imputation for missing data}
\author{\href{https://thestatsgeek.com}{Jonathan Bartlett (thestatsgeek.com)}}
\date{Oslo, November 2019}

\begin{document}
\frame{\titlepage}

\begin{frame}
\tableofcontents[hideallsubsections]
\end{frame}
\begin{frame}{Overview}
\protect\hypertarget{overview}{}

\begin{itemize}
\tightlist
\item
  MI provides valid estimates under the MAR assumption \textbf{and}
  provided the imputation model is reasonably correctly specified.
\item
  If only one variable has missing values, this may be relatively
  simple.
\item
  Commonly though, there may be missing values in many variables, and
  the variables may be a mixture of continuous and discrete, making the
  imputation process more difficult.
\item
  In this session we'll look at the two main approaches to imputation in
  this setting, and some of the important practicalities which arise.
\end{itemize}

\end{frame}

\hypertarget{more-on-mar}{%
\section{More on MAR}\label{more-on-mar}}

\begin{frame}{More on MAR}
\protect\hypertarget{more-on-mar-1}{}

\begin{itemize}
\tightlist
\item
  With one variable partially observed variable \(Y\), the definition of
  MAR is (hopefully) clear - the probability that \(Y\) is missing is
  independent of its value, conditional on other fully observed
  variables \(\mathbf X=(X_{1},X_{2},..,X_{q})\) which are being used in
  the MI analysis.
\item
  Now let's consider the case where the partially observed variable
  \(\mathbf Y=(Y_{1},Y_{2},..,Y_{p})\) consists of multiple components.
\end{itemize}

\end{frame}

\begin{frame}{MAR with monotone missingness}
\protect\hypertarget{mar-with-monotone-missingness}{}

\begin{itemize}
\tightlist
\item
  A special type of missingness pattern is so called monotone
  missingness.
\item
  This means that we can order the components of \(\mathbf Y\) such that
  if \(Y_{j}\) is observed for an observation, all previous components
  are also observed.
\item
  Monotone patterns most commonly occur in longitudinal studies subject
  to dropout:
\end{itemize}

\begin{longtable}[]{@{}lllll@{}}
\toprule
id & 1 & 2 & 3 & 4\tabularnewline
\midrule
\endhead
1 & x & x & x & x\tabularnewline
2 & x & . & . & .\tabularnewline
3 & x & x & . & .\tabularnewline
4 & x & x & x & .\tabularnewline
\bottomrule
\end{longtable}

\end{frame}

\begin{frame}{MAR with monotone missingness}
\protect\hypertarget{mar-with-monotone-missingness-1}{}

\begin{itemize}
\tightlist
\item
  In this case the ordering of variables is on the basis of time (it
  doesn't have to be ordered by time).
\item
  When missingness is monotone, MAR can be shown (see for example
  (Tsiatis \protect\hyperlink{ref-Tsiatis:2006}{2006}) to mean that the
  probability of dropout at time \(t\) doesn't depend on the current or
  future values, conditional on the past (values before \(t\)).
\end{itemize}

\end{frame}

\begin{frame}{MAR with non-monotone missingness}
\protect\hypertarget{mar-with-non-monotone-missingness}{}

\begin{itemize}
\tightlist
\item
  Often in datasets (see the practical) the missingness pattern is not
  monotone.
\item
  Even in longitudinal studies, in addition to dropout, one might have
  intermittent missingness.
\item
  With non-monotone patterns the meaning of MAR becomes more complex -
  the probability of each pattern occurring only depends on the data
  observed under that pattern (Robins and Gill
  \protect\hyperlink{ref-Robins1997}{1997}).
\item
  We will not dwell further on this here, but see the practical for more
  discussion of this issue.
\end{itemize}

\end{frame}

\hypertarget{imputation-from-a-joint-model}{%
\section{Imputation from a joint
model}\label{imputation-from-a-joint-model}}

\begin{frame}[fragile]{Imputation with one partially observed variable}
\protect\hypertarget{imputation-with-one-partially-observed-variable}{}

\begin{itemize}
\tightlist
\item
  In the previous session, we considered a situation where one variable
  \(Y\) has missing values, but other variables \(\mathbf X\) are fully
  observed.
\end{itemize}

\textbf{\(Y\) need not be the outcome in our final model of interest!}

\begin{itemize}
\tightlist
\item
  We used a linear regression model for \(Y|\mathbf X\) to impute the
  missing values of \(Y\). This assumes \(Y|\mathbf X\) is normal, with
  mean a linear function of \(\mathbf X\).
\item
  If \(Y\) were binary, we can similarly use a logistic regression model
  to impute the missing values of \(Y\), given \(\mathbf X\) (e.g.
  \texttt{sen} in the class size study)
\item
  For categorical \(Y\), we can use an ordered or multinomial logistic
  model for \(Y|\mathbf X\).
\end{itemize}

\end{frame}

\begin{frame}{Imputation with multiple partially observed variables}
\protect\hypertarget{imputation-with-multiple-partially-observed-variables}{}

\begin{itemize}
\tightlist
\item
  More typically in epidemiology, we may have more than one variable
  with missing values.
\item
  Extending our previous notation, we now let \(\mathbf Y\) denote the
  vector of variables which have missing values, and let \(\mathbf X\)
  denote the fully observed variables.
\item
  To perform multiple imputation, we must specify a model for the joint
  or multivariate distribution, \(f(\mathbf Y|\mathbf X)\).
\item
  This is tricky in general, particularly if \(\mathbf Y\) contains a
  mixture of continuous and categorical variables.
\end{itemize}

\end{frame}

\begin{frame}{Imputation with multiple partially observed variables}
\protect\hypertarget{imputation-with-multiple-partially-observed-variables-1}{}

\begin{itemize}
\tightlist
\item
  One of the most important multivariate distributions is the
  multivariate normal (MVN).
\item
  MI using the MVN was developed early on.
\end{itemize}

\begin{eqnarray*}
\mathbf Y \sim N(\boldsymbol \mu(\mathbf X), \boldsymbol \Sigma)
\end{eqnarray*}

\begin{itemize}
\item
  For the MVN, each component of \(Y\) is normal, and \(Y_{j}|Y_{-j}\)
  is a linear regression model, where \(Y_{-j}\) denotes all of the
  components of \(\mathbf Y\) except the \(j\)th.
\item
  See (Schafer \protect\hyperlink{ref-Schafer:1997}{1997}) for more
  details.
\end{itemize}

\end{frame}

\begin{frame}{Imputation with the multivariate normal model}
\protect\hypertarget{imputation-with-the-multivariate-normal-model}{}

\begin{itemize}
\tightlist
\item
  What if we have missing values in non-normal, or even categorical
  variables?
\item
  Early advice was that skewed variables could be transformed to
  (approximate) normality before imputation and then back transformed
  afterwards for analysis.
\item
  However, it is important to be aware that if you do this, the
  functional relationships assumed at the imputation stage are changed.
\item
  This has led to one paper recommending against the approach, and
  instead suggesting that if the MVN model is being used to just impute
  ignoring the skewness (Hippel
  \protect\hyperlink{ref-vonHippel2013}{2013}).
\item
  Of course this approach isn't perfect either -- imputing from a
  mis-specified model will in general result in biased inferences, but
  perhaps less biased than if we had used the transformation approach.
\end{itemize}

\end{frame}

\begin{frame}{Imputation with the multivariate normal model}
\protect\hypertarget{imputation-with-the-multivariate-normal-model-1}{}

\begin{itemize}
\tightlist
\item
  For binary variables, some work has been done investigating imputation
  assuming normality, comparing various rounding strategies. See
  (Bernaards, Belin, and Schafer
  \protect\hyperlink{ref-Bernaardsux2fBelin:2007}{2007}) and (Lee and
  Carlin \protect\hyperlink{ref-Leeux2fCarlin:2010}{2010}).
\item
  If the rounding is done carefully, the results can be (somewhat
  surprisingly) quite good.
\item
  However, if \(\mathbf Y\) contains a mixture of continuous and
  categorical variables, using the MVN model is tricky.
\end{itemize}

\end{frame}

\begin{frame}{Joint models for mixtures of continuous and categorical
data}
\protect\hypertarget{joint-models-for-mixtures-of-continuous-and-categorical-data}{}

\begin{itemize}
\tightlist
\item
  Log-linear models were proposed for imputing categorical data.
\item
  With a mixture of continuous and categorical data, the general
  location model was proposed.
\item
  This is available in the R package
  \href{https://cran.r-project.org/package=mix}{mix}.
\item
  These approaches have not however been widely adopted by researchers.
  This may be because for log-linear models one must usually carefully
  choose which interaction parameters to set to zero, and this is quite
  tricky.
\item
  For more details on the model theory, again see (Schafer
  \protect\hyperlink{ref-Schafer:1997}{1997}).
\end{itemize}

\end{frame}

\begin{frame}{Joint models for mixtures of continuous and categorical
data}
\protect\hypertarget{joint-models-for-mixtures-of-continuous-and-categorical-data-1}{}

\begin{itemize}
\tightlist
\item
  Recently there has been further development of more flexible joint
  models for imputation.
\item
  \href{https://cran.r-project.org/package=jomo}{jomo} uses a joint
  model with a latent multivariate normal structure.
\item
  \href{https://cran.r-project.org/package=JointAI}{jointAI} uses a
  joint model factorised as a product of univariate conditional models.
\item
  I expect both to be increasingly used in the coming years in applied
  research.
\end{itemize}

\end{frame}

\hypertarget{imputation-by-chained-equations}{%
\section{Imputation by chained
equations}\label{imputation-by-chained-equations}}

\begin{frame}{Imputation by chained equations / full conditional
specification}
\protect\hypertarget{imputation-by-chained-equations-full-conditional-specification}{}

\begin{itemize}
\tightlist
\item
  Imputation by chained equations (MICE) or full conditional
  specification (FCS) is an alternative to joint model imputation.
\item
  It was proposed independently by (van Buuren, Boshuizen, and Knook
  \protect\hyperlink{ref-VanBuurenux2fBoshuizenux2fKnook:1999}{1999})
  and (Raghunathan et al.
  \protect\hyperlink{ref-Raghunathanux2fLepkowskiux2fVan-Hoewykux2fSolenberger:2001}{2001}).
\item
  Rather than directly specify a joint/multivariate model, we specify a
  series of conditional models.
\end{itemize}

\end{frame}

\begin{frame}{Imputation by chained equations / full conditional
specification}
\protect\hypertarget{imputation-by-chained-equations-full-conditional-specification-1}{}

\begin{itemize}
\tightlist
\item
  e.g.~suppose \(Y_{1}\), \(Y_{2}\) and \(Y_{3}\) have missing values,
  and we have fully observed variables \(\mathbf X\).
\item
  Rather than specify a joint imputation model for
  \(f(Y_{1},Y_{2},Y_{3}|\mathbf X)\) directly, we specify models for:
  \begin{align*}
            f(Y_{1}|Y_{2},Y_{3},\mathbf X) \\
            f(Y_{2}|Y_{1},Y_{3},\mathbf X) \\
            f(Y_{3}|Y_{1},Y_{2},\mathbf X) \\
            \end{align*}
\end{itemize}

\end{frame}

\begin{frame}{MICE/FCS algorithm}
\protect\hypertarget{micefcs-algorithm}{}

For imputation \(m=1,..,M\):

\begin{itemize}
\tightlist
\item
  Initially impute missing values in \(Y_{1}\), \(Y_{2}\) and \(Y_{3}\)
  by randomly sampling from the observed values.
\item
  For iteration \(t=1,..,T\):

  \begin{itemize}
  \tightlist
  \item
    Impute missing values in \(Y_{1}\) \textbf{once} using model for
    \(f(Y_{1}|Y_{2},Y_{3},\mathbf X)\) (using obs. \(Y_{1}\) values and
    observed and imputed values of \(Y_{2}\) and \(Y_{3}\)).
  \item
    Impute missing values in \(Y_{2}\) \textbf{once} using model for
    \(f(Y_{2}|Y_{1},Y_{3},\mathbf X)\) (using obs. \(Y_{2}\) values and
    observed and imputed values of \(Y_{1}\) and \(Y_{3}\)).
  \item
    Impute missing values in \(Y_{3}\) \textbf{once} using model for
    \(f(Y_{3}|Y_{1},Y_{2},\mathbf X)\) (using obs. \(Y_{3}\) values and
    observed and imputed values of \(Y_{1}\) and \(Y_{2}\)).
  \end{itemize}
\item
  Current imputed values of missing values used to form \(m\)th imputed
  dataset.
\end{itemize}

\end{frame}

\begin{frame}{Strengths of MICE/FCS imputation}
\protect\hypertarget{strengths-of-micefcs-imputation}{}

\begin{itemize}
\tightlist
\item
  The major advantage of MICE/FCS imputation (over `joint model'
  imputation) is the ability to specify different model types for each
  variable.
\item
  It has become an extremely popular approach for performing MI (Buuren
  \protect\hyperlink{ref-Buuren:2007}{2007}).
\item
  e.g.~logistic for binary variables, Poisson for count variables (in
  Stata), multinomial logistic for unordered categorical variables.
\item
  It can do things more easily than joint model imputation, such as
  imputing certain variables only in subgroups.
\end{itemize}

\end{frame}

\begin{frame}{Theoretical deficiency of MICE/FCS}
\protect\hypertarget{theoretical-deficiency-of-micefcs}{}

\begin{itemize}
\tightlist
\item
  A theoretical issue with MICE/FCS is that there is no guarantee that
  the algorithm draws imputations from a well defined joint/multivariate
  model.
\item
  Recent work by two groups have identified certain conditions when it
  does (Hughes et al. \protect\hyperlink{ref-Hughes2014}{2014}; Liu et
  al. \protect\hyperlink{ref-Liu2013}{2013}).
\item
  The key condition is that the conditional models are
  \textbf{compatible}.
\item
  This means that there exist multivariate distributions whose
  conditionals are those specified in MICE/FCS.
\item
  Checking compatibility is not easy. In practice, we should be aware of
  the issue, and to situations where incompatibly could seriously
  mislead (see next session).
\end{itemize}

\end{frame}

\begin{frame}{Joint modelling versus MICE/FCS}
\protect\hypertarget{joint-modelling-versus-micefcs}{}

\begin{itemize}
\tightlist
\item
  A number of papers have compared the joint modelling approach with
  chained equations with real examples.
\item
  (Buuren \protect\hyperlink{ref-Buuren:2007}{2007}) applied both
  methods to some growth data, and concluded that chained equations was
  preferable to joint modelling.
\item
  (Lee and Carlin \protect\hyperlink{ref-Leeux2fCarlin:2010}{2010})
  found that both methods worked well in a realistic epidemiological
  setting.
\item
  In settings with continuous and categorical variables with missing
  values, at least in terms of availability of flexible software,
  MICE/FCS seems preferable (in my opinion).
\end{itemize}

\end{frame}

\begin{frame}[fragile]{MICE/FCS with monotone missingness}
\protect\hypertarget{micefcs-with-monotone-missingness}{}

\begin{itemize}
\tightlist
\item
  If the pattern is monotone, there is no need to `cycle' or iterate in
  MICE/FCS.
\item
  This is because one can first impute \(Y2|Y1\), then \(Y3|Y2,Y1\),
  \ldots{}
\item
  Stata's MICE/FCS command \texttt{mi\ impute\ chained} checks for this,
  and if it finds a monotone pattern, imputes sequentially.
\item
  In R, \texttt{mice} has an option (visitSequence) to impute in order
  of increasing missingness (i.e.~the same thing).
\item
  The advantage of this is that we do not need to worry about
  convergence of MICE/FCS.
\item
  We also don't have to worry about incompatibility between the
  conditional models, and the theoretical weakness of MICE/FCS.
\end{itemize}

\end{frame}

\hypertarget{variable-selection-number-of-imputations-and-model-checking}{%
\section{Variable selection, number of imputations, and model
checking}\label{variable-selection-number-of-imputations-and-model-checking}}

\begin{frame}{Which variables should be included in the imputation
model?}
\protect\hypertarget{which-variables-should-be-included-in-the-imputation-model}{}

\begin{itemize}
\tightlist
\item
  Usually, all variables which will be used in our model of interest /
  analysis model should be included in the imputation model.
\item
  In terms of creating the imputations, there is no conceptual
  distinction between variables which are covariates or the outcome in
  your final model of interest.
\item
  If we are imputing missing covariates, the outcome variable
  \textbf{must} be included, to ensure that the imputed covariate values
  have the correct association with the outcome.
\end{itemize}

\end{frame}

\begin{frame}{Auxiliary variables}
\protect\hypertarget{auxiliary-variables}{}

\begin{itemize}
\tightlist
\item
  Often we may have variables \(Z\) which are not involved in our model
  of interest.
\item
  Recall that MI is only valid under MAR, which we cannot verify based
  on the observed data.
\item
  If a variable \(Z\) is predictive of missingness in another variable
  we are imputing, \(Z\) should be included in the imputation model, to
  increase the likelihood that the MAR assumption is satisfied.
\item
  Even if \(Z\) is not predictive of missingness, if it is predictive of
  the partially observed variables \(\mathbf Y\), we should include it
  in the imputation model. Doing so will reduce the uncertainty in
  imputing missing values, thus increasing statistical efficiency.
\end{itemize}

\end{frame}

\begin{frame}{Auxiliary variables}
\protect\hypertarget{auxiliary-variables-1}{}

\begin{itemize}
\tightlist
\item
  The option to include auxiliary variables in the imputation model at
  the imputation stage but omit it from the analysis stage is a big
  advantage of MI
\item
  e.g.~we may have a variable on the causal pathway which we do not want
  to condition on in the analysis
\item
  But we can include it in the imputation stage if we think it will
  improve MAR or is correlated with variables we are imputing
\item
  Some datasets now have many hundreds of variables. In these cases one
  may have to be more judicious in selecting auxiliary variables
\end{itemize}

\end{frame}

\begin{frame}{How many imputations?}
\protect\hypertarget{how-many-imputations}{}

\begin{itemize}
\tightlist
\item
  Earlier papers/books suggested \(M\) could be a small as 3-5
\item
  Validity of inferences is not affected by choice of \(M\)
\item
  Efficiency is improved (somewhat) by increasing \(M\)
\item
  Choose \(M\) so that Monte-Carlo error is sufficiently small \%(see
  Practical for how to assess this in Stata)
\end{itemize}

\end{frame}

\begin{frame}{Model checking}
\protect\hypertarget{model-checking}{}

\begin{itemize}
\tightlist
\item
  For valid inferences we need the imputation models to be correctly
  specified.
\item
  Checking this is not easy.
\item
  One approach, when using MICE/FCS, is to check the fit of each
  conditional model based on its complete case fit.
\item
  Being based on the complete cases, these fits may themselves be
  biased.
\item
  However, we will probably be able to detect and rectify any grossly
  mis-specified models.
\item
  Examining plots of imputed values is also sensible, and can be used to
  diagnose serious issues.
\end{itemize}

\end{frame}

\hypertarget{a-cautionary-example}{%
\section{A cautionary example}\label{a-cautionary-example}}

\begin{frame}{The QRISK study}
\protect\hypertarget{the-qrisk-study}{}

\begin{itemize}
\tightlist
\item
  The QRISK study aimed to derive a new cardiovascular disease (CVD)
  risk score for the UK, based on routinely collected data from general
  practice (Hippisley-Cox et al.
  \protect\hyperlink{ref-Hippisley-Cox2007}{2007}\protect\hyperlink{ref-Hippisley-Cox2007}{a})
\item
  The score was derived using data from 1.28 million patients registered
  at UK GP practices between 1995 and 2007, who were free from CVD at
  registration
\item
  The outcome of interest was time to first recorded diagnosis of CVD
\item
  Cox proportional hazards models were used to model time to CVD, as a
  function of risk factors measured at registration
\end{itemize}

\end{frame}

\begin{frame}[fragile]{Missing data in QRISK}
\protect\hypertarget{missing-data-in-qrisk}{}

\begin{itemize}
\tightlist
\item
  Inevitably there was substantial missingness in `baseline' risk factor
  data
\item
  In particular, 70\% of subjects had HDL cholesterol missing
\item
  The investigators used MI to deal with missing baseline data, using
  the \texttt{ice} (the forerunner to \texttt{mi\ impute\ chained})
  command in Stata
\end{itemize}

\end{frame}

\begin{frame}{Cholesterol and CVD}
\protect\hypertarget{cholesterol-and-cvd}{}

\begin{itemize}
\tightlist
\item
  In the final model, the adjusted hazard ratio for the ratio of total
  to HDL cholesterol was 1.001 (95\% 0.999 to 1.002)
\item
  This suggested that, after adjusting for other baseline risk factors,
  cholesterol had no effect on CVD risk
\item
  Given that cholesterol has been shown to have an independent effect on
  CVD risk in many previous studies, this result was unexpected
\end{itemize}

\end{frame}

\begin{frame}{Cholesterol and CVD}
\protect\hypertarget{cholesterol-and-cvd-1}{}

\begin{itemize}
\tightlist
\item
  A complete case analysis did show evidence for an effect of
  cholesterol
\item
  It turned out that when imputing the missing values, although the time
  to CVD or censoring was included in the imputation model, the
  censoring indicator (1=CVD, 0=censored) had inadvertently not been
  used
\item
  The imputed cholesterol values thus did not have the correct
  association with time to CVD, resulting in there being no evidence of
  an independent effect
\item
  Re-running with a more appropriate imputation model, an independent
  effect of cholesterol was found (Hippisley-Cox et al.
  \protect\hyperlink{ref-Hippisley-Cox2007a}{2007}\protect\hyperlink{ref-Hippisley-Cox2007a}{b})
\end{itemize}

\end{frame}

\begin{frame}{Summary}
\protect\hypertarget{summary}{}

\begin{itemize}
\tightlist
\item
  The meaning of MAR is clear with monotone patterns, but is more
  complex with non-monotone patterns.
\item
  Joint modelling and MICE/FCS are the two broad imputation approaches
  with multivariate data.
\item
  We have discussed practical issues, including number of imputations,
  variable choice, and model checking
\item
  Important to remember: obtaining reasonable results depends on

  \begin{itemize}
  \tightlist
  \item
    the MAR assumption holding (at least approximately)
  \item
    the imputation models used being correctly (at least approximately)
    specified
  \end{itemize}
\end{itemize}

\end{frame}

\begin{frame}[allowframebreaks]{References}
\protect\hypertarget{references}{}

\hypertarget{refs}{}
\leavevmode\hypertarget{ref-Bernaardsux2fBelin:2007}{}%
Bernaards, C A, T R Belin, and J L Schafer. 2007. ``Robustness of a
Multivariate Normal Approximation for Imputation of Incomplete Binary
Data.'' \emph{Statistics in Medicine} 26: 1368--82.

\leavevmode\hypertarget{ref-Buuren:2007}{}%
Buuren, S van. 2007. ``Multiple Imputation of Discrete and Continuous
Data by Fully Conditional Specification.'' \emph{Statistical Methods in
Medical Research} 16: 219--42.

\leavevmode\hypertarget{ref-vonHippel2013}{}%
Hippel, P T von. 2013. ``Should a Normal Imputation Model Be Modified to
Impute Skewed Variables?'' \emph{Sociological Methods \& Research} 42
(1): 105--38.

\leavevmode\hypertarget{ref-Hippisley-Cox2007}{}%
Hippisley-Cox, J, C Coupland, Y Vinogradova, J Robson, M May, and P
Brindle. 2007a. ``Derivation and validation of QRISK, a new
cardiovascular disease risk score for the United Kingdom: prospective
open cohort study.'' \emph{British Medical Journal} 335: 136.

\leavevmode\hypertarget{ref-Hippisley-Cox2007a}{}%
---------. 2007b. ``QRISK authors response {[}electronic response{]}.''
\emph{British Medical Journal} 335: 136.

\leavevmode\hypertarget{ref-Hughes2014}{}%
Hughes, R A, I R White, S R Seaman, J R Carpenter, K Tilling, and J A C
Sterne. 2014. ``Joint Modelling Rationale for Chained Equations.''
\emph{BMC Medical Research Methodology} 14 (1). BioMed Central Ltd: 28.

\leavevmode\hypertarget{ref-Leeux2fCarlin:2010}{}%
Lee, K J, and J B Carlin. 2010. ``Multiple Imputation for Missing Data:
Fully Conditional Specification Versus Multivariate Normal Imputation.''
\emph{American Journal of Epidemiology} 171: 624--32.

\leavevmode\hypertarget{ref-Liu2013}{}%
Liu, J., A. Gelman, J. Hill, Y. S. Su, and J. Kropko. 2013. ``On the
Stationary Distribution of Iterative Imputations.'' \emph{Biometrika}
101: 155--73.

\leavevmode\hypertarget{ref-Raghunathanux2fLepkowskiux2fVan-Hoewykux2fSolenberger:2001}{}%
Raghunathan, Trivellore E, James M Lepkowski, John Van Hoewyk, and Peter
Solenberger. 2001. ``A Multivariate Technique for Multiply Imputing
Missing Values Using a Sequence of Regression Models.'' \emph{Survey
Methodology} 27: 85--95.

\leavevmode\hypertarget{ref-Robins1997}{}%
Robins, J M, and R D Gill. 1997. ``Non-Response Models for the Analysis
of Non-Monotone Ignorable Missing Data.'' \emph{Statistics in Medicine}
16: 39--56.

\leavevmode\hypertarget{ref-Schafer:1997}{}%
Schafer, J L. 1997. \emph{Analysis of incomplete multivariate data}.
London: Chapman; Hall.

\leavevmode\hypertarget{ref-Tsiatis:2006}{}%
Tsiatis, A A. 2006. \emph{Semiparametric Theory and Missing Data}.
Springer, New York.

\leavevmode\hypertarget{ref-VanBuurenux2fBoshuizenux2fKnook:1999}{}%
van Buuren, S., H C Boshuizen, and D L Knook. 1999. ``Multiple
Imputation of Missing Blood Presure Covariates in Survival Analysis.''
\emph{Statistics in Medicine} 18: 681--94.

\end{frame}

\end{document}
